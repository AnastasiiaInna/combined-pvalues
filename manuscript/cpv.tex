\documentclass{bioinfo}
\copyrightyear{2012}
\pubyear{2012}

\begin{document}
\firstpage{1}

\title[comb-p]{Comb-p: software for combining, analyzing, grouping, and correcting spatially correlated p-values}
\author[Pedersen \textit{et~al}]{Brent S. Pedersen\,$^{1,*}$,
Katerina J.  Kechris\,$^{2,*}$,
    Ivana V. Yang\,$^{1}$ and David A. Schwartz\,$^1$}
\address{$^{1}$Department of Medicine, University of Colorado, Denver, Anschutz Medical Campus, Aurora CO 80045, 
USA\\
$^{2}$Department of Statistics, University of Colorado, Denver, Anschutz Medical Campus, Aurora CO 80045, USA\\
}
\history{Received on XXXXX; revised on XXXXX; accepted on XXXXX}
\editor{Associate Editor: XXXXXXX}
\maketitle
\begin{abstract}

\section{Summary:}
\textit{comb-p} is command-line tool and a python library that
manipulates BED files of possibly irregularly spaced p-values and
1) calculates autocorrelation,
2) Combines adjacent p-values,
3) performs false discovery correction,
4) finds regions of enrichment (i.e., series of adjacent low p-values), and
5) assigns significance to those regions.
In addition, tools are provided for visualization and
assessment. The library sets sensible defaults for parameters at each
step based on the structure of the input data, but these may also be customized
by the user. We provide validation and example uses on bisulfite-seq with p-values
from Fisher's exact test,
tiled methylation probes using a linear model, and Dam-ID for chromatin binding
using moderated t-statistics. Because the library
accepts input in a simple, standardized format and is unaffected by the origin
of the p-values, it can be used for a wide variety of applications.

\section{Availability:}
\textit{comb-p} is maintained under the BSD license. The documentation and
implementation are available at

\href{https://github.com/brentp/combined-pvalues}{https://github.com/brentp/combined-pvalues} .
Descriptions of examples and the scripts used to run them are at:

\href{https://github.com/brentp/combined-pvalues/tree/master/examples/}{https://github.com/brentp/combined-pvalues/tree/master/examples/} 
.

\section{Contact:} \href{bpederse@gmail.com}{bpederse@gmail.com},
\href{katerina.kechris@ucdenver.edu}{katerina.kechris@ucdenver.edu}

\end{abstract}

\section{Introduction}
There are a variety of high-throughput technologies that generate
genome-wide data to study processes such as DNA-binding, methylation
status and histone modifications. These technologies include tiling
arrays and sequence-based assays. These high-througput data
are often auto-correlated across
the genome, making inferences difficult, and the significance of
individual regions may be dampened after multiple-testing correction
on potentially millions of sites.
In these studies, hypotheses tests can be performed at each location to
generate p-values for evaluating the effects of interest. For this
purpose, \cite{Kechris2010} developed a method for combining p-values
in sliding windows and accounting for spatial correlations across the
genome. Here, we build on this approach by developing a software tool
that allows for uneven data structure across the genome, more general
auto-correlation calculations, multiple testing corrections for peaks
(i.e., genomic regions of enrichment) and applications to a variety of
different technologies.

\section{Approach}
Many tiling array studies relying on two-sample comparisons are
amenable to the calculation of sliding window averages of log ratios,
t-statistics or other two-sample test statistics. However,
more complex study designs may need to include covariates and report
p-values linear models or other statistical tests.

We utilize a method of calculating 'moving averages' of p-values
that does not depend on the
test used to generate the p-values. \cite{Fisher} developed an
approach of combining p-values from independent tests to get a single
meta-analysis test-statistic with a chi-squared distribution and
degrees of freedom based on the number of tests being combined.  A
similar method developed by \cite{Stouffer} and \cite{Liptak} first
converts p-values to Z-scores which are then summed and scaled to create a single, combined
Z-score.  The Stouffer-Liptak method lends itself well to the addition of
weights on each p-value. \cite{Zaykin2002} introduced a
method to use weights to perform dependence correction on correlated
tests. \cite{Kechris2010} used that framework to create a sliding
window correction where each p-value is adjusted by applying
Stouffer's method to neighboring p-values as weighted according to the
observed auto-correlation at the appropriate distance.

Here, we provide a generic, efficient, and customizable implementation
of these methods with additions including handling variably spaced
probes, a peak finder for dynamically sized regions, and a means to
calculate a p-value for each peak, adjusted for multiple
comparisons. We refer to this implemenation as {\textit comb-p} and
illustrate applications using three different technologies.

\begin{methods}

\section{Implementation}

All programs within \textit{comb-p} expect files in simple BED format
\citep{Kent2002} sorted by chromosome and start.
Additional columns contain the p-value(s) of interest based on the
study design and generated from any software or statistical test.
\textit{comb-p} first
calculates the correlation at varying distance lags (referred to as ACF).
 Whereas Kechris'
\citep{Kechris2010} and many ACF implementations rely on fixed offsets
of adjacent probes, \textit{comb-p} accepts a maxiumum base-pair
distance and a bin size for pairing of probes based on their distance,
so a given probe may have multiple probes in each set of
bins. This is useful in cases where the probes generating the p-values
are unevenly spaced as is common with methylation arrays.

Once the ACF has been calculated, it can be used to perform the
Stouffer-Liptak-Kechris correction (\textit{slk}) where each p-value is
adjusted according to the adjacent p-values and the weight of each of
those neighboring probes is determined from the ACF. The resulting BED file
has an additional column containing the corrected p-value.
A given p-value will be pulled lower if its neighbors also have low
p-values (and little autocorrelation) and likely remain insignificant if the
neighboring p-values are also high.

A q-value score based on the Benjamini-Hochberg false-discovery (FDR)
correction may be calculated. The peak-finding algorithm can then be
used to find enrichment regions or \textit{peaks} on either the FDR
q-value, the \textit{slk}-corrected p-value, or on the original
p-value.  We find that performing the \textit{slk} correction and the
FDR correction before finding peaks is the most stable and
powerful.

The peak-finding works by finding values below a seed, then searching
for neighboring probes below another threshold. A region is extended
as long as it finds nearby probes below the threshold within the given
distance limit. These distance, threshold and seed values are set when
the program is run.

A p-value for each peak can be assigned using the Stouffer-Liptak
correction. This step first calculates the ACF out to a distance
equal to the largest region-size of all predicted peaks. The corrected
p-value for each peak is then calculated using the original p-values
that fall within the peak, and the portion of the ACF that extends to
the distance spanned by that region. Because we use the original,
uncorrected p-values in the calculation of significance for the peak,
we side-step issues of altering the distribution in both the
\textit{slk} and FDR steps. The \textit{region\_p} program reports the
\textit{slk} corrected p-value, and a further one-step \cite{Sidak}
multiple-testing correction accounts for all possible regions of that
size in the genome. For a given region, the number of possible tests
in the Sidak correction is the total bases covered by all input probes
divided by the size of the given region.

\textit{comb-p} is implemented as a single command-line application that
dispatches to multiple independent sub-modules, and as a set of python
packages that may be used programatically. Where possible,
computationally-intensive algorithms are parallelized
automatically--for the ACF and the SLK steps described below--this
results in a speed-up that is linear with the number of cores available.
In all cases the implementation has been tuned to minimize memory-use and
computation time.

When run without arguments, the \textit{comb-p} executable displays the
available programs and a short description of each:
\begin{verbatim}
acf      - autocorrelation within BED file
slk      - Stouffer-Liptak-Kechris correction
fdr      - Benjamini-Hochberg correction
peaks    - find peaks in a BED file.
region_p - SLK p-values for a region
hist     - histogram of a p-values
splot    - scatter plot of a region.
manhattan - a manhattan plot
pipeline  - slk, fdr, peaks, region_p
\end{verbatim}

Though each of these tools may be used independently, the progression of
\textit{acf}, \textit{slk}, \textit{fdr}, \textit{peaks}, \textit{region\_p}
described below works well for the examples presented. These steps can
be run successively with the single command: \textit{pipeline}.

\section{Application}
We extend 3 previously published analyses to demonstrate the utility and
breadth of use of \textit{comb-p} and to validate against the
published data. The full set of commands used to run each of these analyses
is availalable in the corresponding subdirectory at:
https://github.com/brentp/combined-pvalues/tree/master/examples

Comprehensive, High-throughput Array for Relative Methylation (CHARM)
is a tiling array used to quantify methylation at CpG-rich sites
\citep{Irizarry2008}.  Adjacent probes are spatially auto-correlated
due to the regional nature of methylation, and to the overlapping
probes on the tiling array.  The CHARM study reported in
\cite{Irizarry2009} contains probe data for tumor and normal samples
from a variety of tissues. Using this data, we fit a linear model in R
\citep{R} to
obtain p-values for the significance of tumor or tissue-specific
effects on methylation status at each probe.  We then run
\textit{comp-p}, which finds 85\% of the tumor-specific differentially
methylated regions (DMR's) reported by \cite{Irizarry2009} and 75\% of
the tissue-specific DMR's reported.  While these overlaps depend on
the chosen parameters, we show that the Irizarry DMR's that overlap
with the \textit{comb-p} DMR's have signficantly (t-test,
p=1.296e-145) lower FDR values (as reported by Irizarry) than those
that do not, indicating that \textit{comb-p} is finding the most
significant DMR's.

Bisulfite-sequencing is also used to measure methylation across the
genome. The BS data
has counts for methylated and un-methylated cytosines.  We
analyze data described in \cite{Hsieh2009} from {\it Arabidopsis
thaliana} using MethylCoder \citep{Pedersen2011} to map the
bisulfite treated reads to the genome. At each location,
we use a Fisher's exact test to obtain p-values for the counts of
converted and un-coverted Cytosines between endosperm and embryo. We find
differentially methylated regions between these two tissues associated
with genes enriched for gene ontologies related to the ribosome (p=
1e-3)[TODO: Ivana ribosome related to growth or something?].

DamID technology followed by tiling arrays can be used to map genomic
regions bound by DNA-binding proteins \citep{Steensel2001}.  We
recreate an analysis of the DamID tiling array data for the Ci
transcription factor in {\it Drosophila melanogaster} \citep{Biehs}.
\cite{Kechris2010} calculated the
mean log ratios of intensities between Ci experimental and control
samples and used a one-sided moderated t-test \citep{Limma2005} to obtain
p-values to test each probe for positive intensity.
Of the original 878 regions, 695 (79\%) are
represented in the \textit{comb-p} predicted peaks.  To score each
peak, the authors report $-\log_{10}$ of the smallest raw p-value for
each peak. The published peaks overlappng the \textit{comb-p} regions
have a higher score than those that do not (t-test, p=2.98e-59),
indicating that \textit{comb-p}
is finding the best regions among those previously reported.

\end{methods}

\section{Conclusion}

The \textit{comb-p} software is useful in contexts where p-values are
generated across the genome and where there is autocorrelation among
nearby p-values. We have outlined our implementation and demonstrated
the utility on data from three different technologies each from a
different statistical test. We have also validated our method using
previously published results from those same studies.

\section*{Acknowledgement}

\paragraph{Funding\textcolon} This work was supported by grant from NIH to KK (AA016922) and  ???.

%\bibliographystyle{natbib}
%\bibliographystyle{achemnat}
%\bibliographystyle{plainnat}
%\bibliographystyle{abbrv}
%\bibliographystyle{bioinformatics}
%\bibliographystyle{plain}
%\bibliography{Document}

\begin{thebibliography}{}

\bibitem[Kechris \textit{et~al}., 2010]{Kechris2010}
Kechris, K.J. et al. (2010)
Generalizing moving averages for tiling arrays using combined p-value
statistic. {\it Statistical Applications in Genetics and Molecular Biology}
{\bf 9}, Article 29.

\bibitem[Kent \textit{et al.}, 2002]{Kent2002} Kent, W.J. et al. (2002) The human genome browser at UCSC. {\it 
Genome Res.}, {\bf 12}, 996-1106.

\bibitem[Sidak, 1967]{Sidak}
Sidàk, Z. (1967).
Rectangular confidence region for the means of multivariate normal distributions.
\textit{Journal of the American Statistical Association}, {\bf 62}, 626-633.

\bibitem[Fisher, 1948]{Fisher}
Fisher, R.A. (1948)
Questions and answers \#14.
{\it The American Statistician}, {\bf 2}(5), 30-31.

\bibitem[Stouffer \textit{et al.}, 1949]{Stouffer}
Stouffer, S.A. \textit{et~al.} (1949). \textit{The American Soldier},
Princeton University Press, Princeton, NJ. Vol.1: Adjustment during Army Life.

\bibitem[Liptak, 1958]{Liptak}
Liptak, T. (1958). On the combination of independent tests. {\it Magyar Tud.
Akad. Mat. Kutato Int. Kozl.}, {\bf 3}, 171-197.

\bibitem[Zaykin {\it et~al}., 2002]{Zaykin2002} Zaykin, D.L. et al. (2002).
Truncated product method for combining p-values. {\it Genetic Epidemiology},
{\bf 22}, 170-185.

\bibitem[Irizarry {\it et~al}., 2008]{Irizarry2008} Irizarry, R.A. et al.
(2008). Comprehensive high-throughput arrays for relative methylation (CHARM).
{\it Genome Research}, {\bf 18}, 780-790.

\bibitem[Irizarry {\it et~al}., 2009]{Irizarry2009} Irizarry, R.A. et al.
(2009). The human colon cancer methylome shows similar hypo- and
hypermethylation at conserved tissue-specific CpG island shores.
{\it Nature Genetics}, {\bf 41}, 178-186.

\bibitem[Pedersen {\it et~al}., 2011]{Pedersen2011} Pedersen, B.S. et al.
(2011). MethylCoder: software pipeline for bisulfite-treated sequences.
{\it Bioinformatics}, {\bf 27}, 2435-2436.

\bibitem[van Steensel {\it et~al}., 2001]{Steensel2001}van Steensel, B., Delrow, J. and Henikoff, S. (2001):
Chromatin Profiling Using Targeted DNA Adenine Methyltransferase,
{\it Nature Genetics}, {\bf 27}, 304-308.

\bibitem[Hsieh {\it et~al}., 2009]{Hsieh2009} Hsieh, T.F. et al. (2009).
Genome-Wide Demethylation of Arabidopsis Endosperm.
{\it Science}, {\bf 324}, 1451-1454.

\bibitem[Smyth, 2005]{Limma2005}
Smyth, G. K. (2005). Limma: linear models for microarray data. In: Bioinformatics and Computational Biology Solutions using R and Bioconductor, R. Gentleman, V. Carey, S. Dudoit, R. Irizarry, W. Huber (eds.), Springer, New York, 397-420.

\bibitem[Biehs {\it et~al}, 2010]{Biehs}
B. Biehs, K. Kechris, S. Liu and T. Kornberg (2010).
Hedgehog targets in the Drosophila embryo and the mechanisms that generate tissue-specific outputs of Hedgehog signaling. \it{Development}, {\bf 137}, 3887-3898.

\bibitem[R project, 2005]{R}
R Development Core Team R. (2005).
A Language and Environment for Statistical Computing.
R Foundation for Statistical Computing; Vienna, Austria: 2005. URL http://www.R-project.org.


\end{thebibliography}
\end{document}
